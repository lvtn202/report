Trong phạm vi báo cáo này, mục tiêu của nhóm nghiên cứu là xây dựng hệ thống quản lý quá trình gia công nhuộm cho một doanh nghiệp vải sợ trên nền web, chưa tập trung vào việc tối ưu, mở rộng cũng như tích hợp nhiều hệ thống khác. Tuy nhiên, trong tương lai, nhóm nghiên cứu đề xuất một số cải tiến để hệ thống hoàn thiện hơn và có thể trở thành
ứng dụng rộng rãi với nhiều tính năng mới hữu ích.\par

Một số hướng phát triển trong tương lai:
\begin{itemize}
    \item Tối ưu hóa mã nguồn, cả về front-end và back-end. Cải thiện hiệu năng, giảm thời gian phản hồi của các chức năng khi phải làm việc với khói dữ liệu lớn dần trong quá trình sử dụng hệ thống.
    \item Tích hợp các thiết bị hỗ trợ cho việc nhập và xuất dữ liệu dễ dàng hơn: máy quét mã vạch cho cây vải, máy in để in thông tin vải nhuộm, công nợ, thanh toán,...
    \item Hệ thống hiện tại có các loại user cơ bản là admin và nhân viên, sẽ có nhiều loại user trong tương lai. Việc mở rộng user sẽ khá dễ dàng do kiến trúc hệ thống được xây dựng có khả năng hỗ trợ mở rộng user.
    \item Để giảm cường độ request cũng như thời gian xử lí của backend, đồng thời giảm nguy cơ tắt nghẽn đường truyền và khả năng chống chịu của hệ thống, trong tương lai hệ thống có thể được nâng cấp thành micro-service.
    \item Hệ thống mới quản lí quá trình gia công nhuộm, chưa quản lí được quá trình sản xuất vải sợi của xưởng. Trong tương lai, đó là hướng phát triển mới.
    \item Đưa hệ thống vào sử dụng trong thực tế. Đó là mục đích cuối cùng nhất của đề tài Luận văn tốt nghiệp này.
\end{itemize}
