Doanh nghiệp sản xuất vải sợi có nhu cầu số hóa quy trình quản lí kinh doanh, cụ thể là số hóa quá trình gia công nhuộm. Doanh nghiệp sản xuất vải sợi, nhưng không gia công nhuộm cho vải sợi của mình. Vải sợi khi được sản xuất ra phải được gia công nhuộm mới có thể đưa ra thị trường, vì vậy doanh nghiệp sẽ đặt hàng các xưởng nhuộm bên ngoài. Từ trước tới nay, việc quản lý quá trình đặt hàng các xưởng nhuộm đều thực hiện trên giấy tờ thủ công, điều này tốn rất nhiều thời gian, công sức cũng như không hoàn toàn đảm bảo độ chính xác của số liệu.\par
Nhiệm vụ của nhóm thực hiện đề tài là xây dựng hệ thống quản lý quá trình gia công nhuộm này trên nền web. Ứng dụng có thể quản lý:
\begin{itemize}
    \item Xưởng nhuộm: danh sách, thông tin xưởng, hàng tồn, hàng thành phẩm, công nợ và thanh toán.
    \item Vải mộc: danh sách, số lượng, độ dài, loại vải, xuất vải thô cho các xưởng.
    \item Vải thành phẩm: danh sách, số lượng, độ dài, loại vải, màu sắc, nhập vải thành phẩm từ các xưởng.
    \item Đơn đặt hàng: danh sách, thông tin đơn, tạo đơn, cập nhật trạng thái đơn.
    \item Hàng trả: danh sách, thông tin hàng trả, tạo hàng trả.
    \item Quản lí người dùng.
\end{itemize}
Để hoàn thành đề tài này, nhóm đã thực hiện những công việc sau:
\begin{itemize}
    \item Tìm hiểu quy trình đặt hàng xưởng nhuộm bên ngoài, bao gồm mô hình kinh doanh, nội dung dữ liệu và các luồng dữ liệu.
    \item Tìm hiểu khiến thức nền tảng công nghệ cần thiết để xây dựng hệ thống trên nền web.
    \item Thực hiện từng bước các công việc theo sự hướng dẫn của giảng viên hướng dẫn.
    \item Hoàn thành ứng dụng, kiểm tra ứng dụng, tối ưu hóa ứng dụng.
    \item Tự xem xét đánh giá những gì nhóm đã thực hiện, những ưu điểm và nhược điểm, hướng phát triển trong tương lai.
    \item Hoàn thành báo cáo.
\end{itemize}
