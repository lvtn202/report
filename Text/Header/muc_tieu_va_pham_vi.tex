\subsection{Mục tiêu đề tài}

Mục tiêu của đề tài là xây dựng một ứng dụng trên nền web giúp doanh nghiệp sản xuất vải sợi có thể thực hiện được các nghiệp vụ quản lý quá trình gia công nhuộm của mình. Việc này đòi hỏi nhóm phải tìm hiểu quy trình nghiệp vụ gia công nhuộm để có thể hiện thực đúng các yêu cầu đầu vào và đầu ra mà doanh nghiệp mong muốn, từ lúc xuất cây vải mộc cho xưởng nhuộm, trải qua quá trình gia công nhuộm, nhập cây vải thành phẩm về doanh nghiệp, quản lí thanh toán,  và quản lí hàng trả. Việc có một hệ thống như thế sẽ giúp doanh nghiệp không gặp sai sót khi tính toán thủ công, giảm thiểu được nhân lực cho doanh nghiệp, ngoài ra còn có các bảng biểu thống kê trực quan giúp doanh nghiệp có thể xem được hiệu năng của các xưởng nhuộm để từ đó chọn ra xưởng nhuộm nào phù hợp với doanh nghiệp và có thể hợp tác lâu dài. \par

Việc thực hiện đề tài này cũng giúp nhóm thực hiện có cơ hội ôn tập lại và áp dụng các kiến thức mình đã được học trong chương trình đại học: các giải thuật, quy trình phát triển phần mềm, thiết kế cơ sở dữ liệu,... và áp dụng các công nghệ đang phổ biến hiện nay và dự án của mình. \par

\subsection{Phạm vi đề tài}

Hệ thống được hiện thực với chức năng quản lí quá trình gia công nhuộm ở các xưởng nhuộm, do đó các chức năng chỉ quản lí bắt đầu từ việc xưởng xuất các cây vải mộc đến các xưởng nhuộm, trải qua quá trình gia công nhuộm sẽ tạo ra cây vải thành phẩm và nhập về doanh nghiệp, và các chức năng liên quan gồm có quản lí đơn đặt hàng, quản lí công nợ của xưởng, thanh toán công nợ và quản lí việc trả các cây hàng bị lỗi. \par

Các nghiệp vụ liên quan đến quản lí kho và quản lí việc mua bán vải thành phẩm không nằm trong phạm vi của đề tài. Do đó việc quản lí các cây vải mộc ở công ty trước khi xuất ra xưởng nhuộm và các cây vải thành phẩm sau khi nhập từ xưởng nhuộm về sẽ không được hiện thực trong đề tài này. \par

Người dùng của hệ thống chỉ là các nhân viên trong doanh nghiệp, không bao gồm nhân viên ở các xưởng nhuộm và doanh nghiệp vận chuyển vải. \par