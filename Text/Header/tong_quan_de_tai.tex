Trong nền công nghiệp hiện đại 4.0 hiện nay, các doanh nghiệp hoạt động với dây chuyền sản xuất thực hiện thủ công và quản lí doanh nghiệp thông qua sổ sách viết tay có thể sẽ bị quá tải với khối lượng công việc và hàng hóa khổng lồ, khó bắt kịp các đối thủ của mình. Để tiếp tục phát triển và cạnh tranh, các doanh nghiệp đòi hỏi phải thay đổi quy trình hoạt động và quản lí của mình từ thủ công sang các hệ thống tự động, sử dụng các phần mềm quản lí chuyên dụng. \par

Hiện nay, có nhiều doanh nghiệp lựa chọn số hóa toàn bộ quy trình hoạt động của mình, bên cạnh đó cũng có doanh nghiệp lựa chọn số hóa một phần quy trình hoạt động, ưu tiên một số dây chuyền quan trọng trước, vừa phù hợp với kinh phí của doanh nghiệp nhưng cũng có thể đảm bảo có thể cạnh tranh với các doanh nghiệp đối thủ. \par

Với tinh thần trên, nhóm thực hiện quyết định xây dựng một hệ thống "Quản lý quá trình gia công nhuộm", nhằm mục đích đáp ứng một số nhu cầu của các doanh nghiệp sản xuất vải sợi bao gồm: quản lý các cây vải mộc của doanh nghiệp, quản lý các cây vải thành phẩm sau khi nhuộm, quản lý các cây vải đang tồn kho ở các xưởng gia công, quản lý công nợ của các xưởng gia công, và quản lý các cây vải lỗi trả lại. \par

Dự án được thực hiện bởi nhóm sinh viên tại trường Đại học Bách Khoa thành phố Hồ Chí Minh.  \par