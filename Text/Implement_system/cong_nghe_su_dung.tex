Để xây dựng được một hệ thống lớn như vậy, nhóm đã sử dụng rất nhiều thư viện, framework như đã được giới thiệu trong Phần 2. Các công nghệ nổi bật được liệt kê trong [Bảng \ref{usedtech}]. Sở dĩ nhóm lựa chọn các công nghệ này là do đây là các công nghệ phổ biến có số lượng người dùng đông đảo, cộng đồng hỗ trợ lớn, và dễ tiếp cận đối các các dự án làm trong thời gian ngắn như Luận văn tốt nghiệp.

\begin{table}[H]
    \centering
    \begin{tabular}{|m{6cm}|m{3cm}|m{6cm}|}
    \hline 
        \textbf{Thư viện, framework} & \textbf{Phiên bản} & \textbf{Chức năng}\\ \hline
        React & 17.0.2 & Xây dựng frontend \\ \hline
        Recharts & 2.0.9 & Hỗ trợ hiện thực các biểu đồ thống kê \\ \hline
        Axios & 0.21.1 & Hỗ trợ gọi HTTP Request đến server \\ \hline
        react-redux & 7.2.2 & Hỗ trợ quản lí trạng thái ứng dụng \\ \hline
        redux-thunk & 2.3.0 & Xây dựng middlewares cho ứng dụng \\ \hline
        react-router-dom & 5.2.0 & Quản lí việc chuyển trang web \\ \hline
        @material-ui/core & 4.11.3 & Cung cấp các UI component \\ \hline
        Java & 1.8 & Môi trường runtime Java \\ \hline
        Spring & 2.4.3 & Xây dựng backend \\ \hline
        Maven & 3.6.3 & Quản lí thư viện \\ \hline
        Query DSL & 3.6.8 & Xây dựng các câu truy vấn \\ \hline
        JDBC & 2.4.3 & Kết nối database \\ \hline
        Lombok & 1.18.18 & Hỗ trợ sinh code tự động \\ \hline
        Postgre SQL & 13.3 & Hệ quản trị cơ sở dữ liệu \\ \hline
    \end{tabular}
    \caption{Công nghệ sử dụng.}
    \label{usedtech}
\end{table}